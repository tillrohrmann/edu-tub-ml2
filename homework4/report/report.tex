\documentclass[a4paper, 12pt, titlepage]{article}

% Including needed packages
\usepackage[margin=2cm]{geometry}
\usepackage{amsmath}
\usepackage{amssymb}
\usepackage{amsthm}
\usepackage{graphicx}
\usepackage{subfig}
\usepackage{float}


\DeclareMathOperator{\tr}{tr}

\title
{{\em Machine learning 2}\\
Exercise sheet 4}
\author{FLEISCHMANN Kay, Matrnr: 352247\\
	ROHRMANN Till, Matrnr: 343756}
\date{\today}

\begin{document}

\maketitle

\setcounter{section}{5}


\section{Kernel Canonical Correlation Analysis}
(a) Given some trainingdata $X \in \mathbb{R}^{d1 \times N} $ and $Y \in \mathbb{R}^{d2 \times N}$. The idea behind CCA is to find two one dimensional projections  $w_x$ and $w_y$ into  space in which a latent variable exists with maximal correlation. Let $C_{xx} = XX^T$, $C_{yy} = YY^T$, $C_{xy} = XY^T$ and $C_{yx} = YX^T$.
\newline \newline
Formally, Find $w_x \in \mathbb{R}^{d1}$, $w_y \in \mathbb{R}^{d2}$ which
\begin{eqnarray}
  maximaize & w^{T}_{x} C_{xy}w_y
 \end{eqnarray}
with subject to
\begin{eqnarray}
  w^T_xC_{xx}w_x=1 \\
  w^T_yC_{yy}w_y=1 
\end{eqnarray}
Show that it is always possible to find an optimal solution in the span of the data, that is $w_x= X \alpha_x$, $w_y = Y \alpha_y$:
\begin{eqnarray}
TODO ??
\end{eqnarray}
Derive the dual optimization problem: \newline
\begin{eqnarray}
  \mathcal{L}(\alpha, \beta) &=& w^{T}_{x} C_{xy}w_y - \frac{1}{2} \alpha (w^T_xC_{xx}w_x-1) \frac{1}{2} \beta (w^T_yC_{yy}w_y-1)
  \end{eqnarray}
\begin{eqnarray}
  \frac{\partial L}{\partial w^T_x} = XY^Tw_y - \alpha( XX^Tw_x ) = 0  \Leftrightarrow XY^Tw_y = \alpha( XX^Tw_x ) \\
  \frac{\partial L}{\partial w^T_y} = XY^Tw_x - \beta ( YY^Tw_y) = 0 \Leftrightarrow  XY^Tw_x = \beta ( YY^Tw_y)
\end{eqnarray}
Multiplication $w^T_x$ with Equation (6) and $w^T_y$ with Equation (7) results to:
\begin{eqnarray}
  w^T_xXY^Tw_y = \alpha( w^T_xXX^Tw_x ) \\
  w^T_yXY^Tw_x = \beta ( w^T_yYY^Tw_y) 
\end{eqnarray}
Because of constraints (1) and (2) 
\begin{eqnarray}
  \alpha( XX^Tw_x )=\beta ( w^T_yYY^Tw_y)  \Rightarrow \alpha = \beta
\end{eqnarray}
Next we combine equation (6)-(10)
\begin{eqnarray}
  C_{xy}w_y = \alpha C_{xx}  w_x \\
  C_{yx}w_x = \alpha C_{yy}  w_y
\end{eqnarray}
and witen in a matrix we get finally an eigenvalue problem: \newline \newline
$
\begin{bmatrix} 0 & C_{xy} \\ C_{yx} & 0 \end{bmatrix}
\begin{bmatrix} w_x \\ w_y \end{bmatrix} =  \alpha
\begin{bmatrix} C_{xx} & 0 \\ 0 & C_{yy} \end{bmatrix}
\begin{bmatrix} w_x \\ w_y \end{bmatrix}
$
\newline \newline

\subsection{Kernel}
With the assumtion of (4) and the equations (1)-(3) and $w_x= X \alpha_x$, $w_y = Y \alpha_y$:

\begin{eqnarray}
  maximize & \alpha^T_xX^TXY^TY\alpha_y = \alpha^T_xK_xK_y\alpha_y
 \end{eqnarray}
with subject to
\begin{eqnarray}
  \alpha^T_xX^TXX^TX\alpha_x = \alpha^T_xK_xK_x\alpha_x = 1\\
  \alpha^T_yY^TYY^TY\alpha_y = \alpha^T_yK_yK_y\alpha_y = 1\\
\end{eqnarray}

Put this into the matrix formulation as followed in (12) \newline
Kernel-equation:
\newline
$
\begin{bmatrix} 0 & K_xK_y \\ K_yK_x & 0 \end{bmatrix}
\begin{bmatrix} \alpha_x \\ \alpha_y \end{bmatrix} =  \rho
\begin{bmatrix} K^2_x & 0 \\ 0 & K^2_y \end{bmatrix}
\begin{bmatrix} \alpha_x \\ \alpha_y \end{bmatrix}
$ \newline \newline
(b) \textit{ Describe how the generalized eiganvalue problem from exercise (a) - and thun CCA - can be kernalized. }
Computing the covariance of $X^TX$ or $Y^TY$ for realy big matrices can be very expensive. The Kernel-Equation shows, that the solution of the generalized eigenvalue problem just depends on the scalar product between the datapoints. Therefore knowlege  about the kernel (this means the scalarproduct between points $X^TX$ or $Y^TY$) is enough to compute valid solutions.



\end{document}
